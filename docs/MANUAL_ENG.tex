\documentclass[]{book}
\usepackage{lmodern}
\usepackage{amssymb,amsmath}
\usepackage{ifxetex,ifluatex}
\usepackage{fixltx2e} % provides \textsubscript
\ifnum 0\ifxetex 1\fi\ifluatex 1\fi=0 % if pdftex
  \usepackage[T1]{fontenc}
  \usepackage[utf8]{inputenc}
\else % if luatex or xelatex
  \ifxetex
    \usepackage{mathspec}
  \else
    \usepackage{fontspec}
  \fi
  \defaultfontfeatures{Ligatures=TeX,Scale=MatchLowercase}
\fi
% use upquote if available, for straight quotes in verbatim environments
\IfFileExists{upquote.sty}{\usepackage{upquote}}{}
% use microtype if available
\IfFileExists{microtype.sty}{%
\usepackage{microtype}
\UseMicrotypeSet[protrusion]{basicmath} % disable protrusion for tt fonts
}{}
\usepackage{hyperref}
\hypersetup{unicode=true,
            pdftitle={Un petit manual de psychologie contemporaine à destination de ceux qui souffrent},
            pdfauthor={Arnaud Passé},
            pdfborder={0 0 0},
            breaklinks=true}
\urlstyle{same}  % don't use monospace font for urls
\usepackage{longtable,booktabs}
\usepackage{graphicx,grffile}
\makeatletter
\def\maxwidth{\ifdim\Gin@nat@width>\linewidth\linewidth\else\Gin@nat@width\fi}
\def\maxheight{\ifdim\Gin@nat@height>\textheight\textheight\else\Gin@nat@height\fi}
\makeatother
% Scale images if necessary, so that they will not overflow the page
% margins by default, and it is still possible to overwrite the defaults
% using explicit options in \includegraphics[width, height, ...]{}
\setkeys{Gin}{width=\maxwidth,height=\maxheight,keepaspectratio}
\IfFileExists{parskip.sty}{%
\usepackage{parskip}
}{% else
\setlength{\parindent}{0pt}
\setlength{\parskip}{6pt plus 2pt minus 1pt}
}
\setlength{\emergencystretch}{3em}  % prevent overfull lines
\providecommand{\tightlist}{%
  \setlength{\itemsep}{0pt}\setlength{\parskip}{0pt}}
\setcounter{secnumdepth}{5}
% Redefines (sub)paragraphs to behave more like sections
\ifx\paragraph\undefined\else
\let\oldparagraph\paragraph
\renewcommand{\paragraph}[1]{\oldparagraph{#1}\mbox{}}
\fi
\ifx\subparagraph\undefined\else
\let\oldsubparagraph\subparagraph
\renewcommand{\subparagraph}[1]{\oldsubparagraph{#1}\mbox{}}
\fi

%%% Use protect on footnotes to avoid problems with footnotes in titles
\let\rmarkdownfootnote\footnote%
\def\footnote{\protect\rmarkdownfootnote}

%%% Change title format to be more compact
\usepackage{titling}

% Create subtitle command for use in maketitle
\providecommand{\subtitle}[1]{
  \posttitle{
    \begin{center}\large#1\end{center}
    }
}

\setlength{\droptitle}{-2em}

  \title{Un petit manual de psychologie contemporaine à destination de ceux qui souffrent}
    \pretitle{\vspace{\droptitle}\centering\huge}
  \posttitle{\par}
    \author{Arnaud Passé}
    \preauthor{\centering\large\emph}
  \postauthor{\par}
      \predate{\centering\large\emph}
  \postdate{\par}
    \date{2020-01-04}


\begin{document}
\maketitle

{
\setcounter{tocdepth}{0}
\tableofcontents
}
\hypertarget{introduction}{%
\chapter*{Introduction}\label{introduction}}
\addcontentsline{toc}{chapter}{Introduction}

Ce livre est une série de notes prises pour documenter ma progression dans la compréhension des enseignements de la psychologie contemporaine, notamment pour ce qui a trait à la résolution des souffrances psychologiques.

Les souffrances psychologiques sont une des sources majeures de mal-être dans les sociétés développées contemporaines.
Je développe dans mes romans des exemples de la souffrance psychologique et de ses résultats sur la vie des gens.
Dans ce petit manuel, je compile ce que j'ai compris et ce que j'ai trouvé utile dans les enseignement de la psychologie contemporaine pour résoudre les souffrances psychologiques.

J'aborde notamment les approches comportementale et cognitive, la théorie de la personnalité et la science de la mesure du bien-être ainsi que les programmes de développement des compétences socio-émotionnelles chez les enfants.

Ce livre vise à familiariser les lecteurs français avec les résultats de la psychologie comportementale et cognitive et en donner ma propre lecture.
Un livre peut être une source de réconfort et de croissance personnelle, j'en ai fait maintes fois l'expérience.
C'est notamment le cas si vous souhaitez améliorer votre confort de vie et réduire votre souffrance psychologique alors qu'elle n'est pas pathologique.

Si vous souffrez dans des conditions bien supérieures (et plusieurs tests dans le livre vous aideront à le décider), consultez.
Un livre ne saurait se substituer à une relation thérapeutique.

Ce livre est aussi open source et collaboratif, donc si vous le souhaitez, vous pouvez le brancher et faire des corrections et des propositions.

\hypertarget{measure}{%
\chapter{Measure}\label{measure}}

\hypertarget{personality-traits}{%
\section{Personality traits}\label{personality-traits}}

\hypertarget{iq}{%
\section{IQ}\label{iq}}

\hypertarget{welfare}{%
\section{Welfare}\label{welfare}}

\hypertarget{psychological-suffering}{%
\section{Psychological suffering}\label{psychological-suffering}}

\hypertarget{correlate}{%
\chapter{Correlate}\label{correlate}}

\hypertarget{big5-and-life-events}{%
\section{Big5 and life events}\label{big5-and-life-events}}

\hypertarget{big5-and-welfare}{%
\section{Big5 and welfare}\label{big5-and-welfare}}

\hypertarget{big5-and-psychological-disease}{%
\section{Big5 and psychological disease}\label{big5-and-psychological-disease}}

\hypertarget{cure}{%
\chapter{Cure}\label{cure}}

\hypertarget{rational-therapy}{%
\section{Rational therapy}\label{rational-therapy}}

\hypertarget{tcc}{%
\section{TCC}\label{tcc}}

\hypertarget{third-generation-therapies}{%
\section{Third generation therapies}\label{third-generation-therapies}}

\hypertarget{programs-for-children}{%
\section{Programs for Children}\label{programs-for-children}}

\hypertarget{practice}{%
\chapter{Practice}\label{practice}}


\end{document}
